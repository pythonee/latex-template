% !Mode:: "TeX:UTF-8"

\documentclass[xcolor=dvipsnames,CJK,hyperref={bookmarks=false}]{beamer}
\usepackage{py-talk}

\begin{document}
\begin{CJK*}{UTF8}{hei}
\title{\Huge{演讲标题}}
\subtitle{副标题}
\author{主讲人:王飞}
\date{\today}
\institute{公司}
\begin{frame}
\titlepage
\end{frame}

\section*{今天主题大纲}
\begin{frame}
%\begin{multicols}{2}
  \setcounter{tocdepth}{1}
  \tableofcontents
%\end{multicols}
\end{frame}
\section{第一个主题}
\subsection{第一个小主题}
\begin{frame}
中文
\end{frame}
\subsection{第二个小主题}
\begin{frame}Test\end{frame}

\section{第二个主题}
\begin{frame}
中文
\end{frame}

\begin{frame}[fragile]
\begin{example}[Putting Verbatim]
\begin{verbatim}
\begin{document}

\end{document}
\end{verbatim}
\end{example}
\end{frame}

\section{列举}
\begin{frame}
  \begin{block}{Ubuntu with TeX Live}
    \begin{enumerate}
      \item Place the {\tt <dirstruct>} in the root of your local latex directory tree. By default it is\\
        {\tt \textasciitilde /texmf}\\
        If the root does not exist, create it. The symbol {\tt \textasciitilde} refers to your home folder, i.e., {\tt /home/<username>}
      \item In a terminal run\\
        {\tt \$ texhash \textasciitilde /texmf}
    \end{enumerate}
  \end{block}
\end{frame}

\section{块结构}

\begin{frame}

\begin{block}{title of the bloc}
bloc text
\end{block}

\begin{exampleblock}{title of the bloc}
bloc text
\end{exampleblock}

\begin{alertblock}{title of the bloc}
bloc text
\end{alertblock}
\end{frame}

\section{表格}
\begin{frame}
\begin{tabular}{c c c}
A & B & C \\
\pause
1 & 2 & 3 \\
\pause
A & B & C \\
\end{tabular}
\end{frame}

\section{插入图片}
  \frame
  {

    A graphics file is included by using the {\bf figure} environment,
    and inside of that the {\bf includegraphics} command.
    \begin{figure}
      \scalebox{0.50}
      {
        \includegraphics{pengbrew.png}
      }
    \end{figure}
  }

\section{定理}
\begin{frame}
\begin{theorem}
The quick brown fox jumps over the lazy dog.
\end{theorem}
\end{frame}

\end{CJK*}
\end{document} 