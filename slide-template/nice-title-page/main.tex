% !Mode:: "TeX:UTF-8"

\documentclass[hyperref={bookmarks=false},CJK,ucs,9pt]{beamer}

% Copyright 2004 by Till Tantau <tantau@users.sourceforge.net>.
%
% In principle, this file can be redistributed and/or modified under
% the terms of the GNU Public License, version 2.
%
% However, this file is supposed to be a template to be modified
% for your own needs. For this reason, if you use this file as a
% template and not specifically distribute it as part of a another
% package/program, I grant the extra permission to freely copy and
% modify this file as you see fit and even to delete this copyright
% notice.
%
% Modified by Tobias G. Pfeiffer <tobias.pfeiffer@math.fu-berlin.de>
% to show usage of some features specific to the FU Berlin template.

% remove this line and the "ucs" option to the documentclass when your editor is not utf8-capable
\usepackage[utf8x]{inputenc}    % to make utf-8 input possible
\usepackage[english]{babel}     % hyphenation etc., alternatively use 'german' as parameter
\usepackage{xltxtra,fontenc,xunicode}
\usepackage[slantfont,boldfont]{xeCJK}

\setCJKmainfont{Adobe Song Std L}
\setCJKsansfont{Adobe Heiti Std}
\setCJKmonofont{SimSun}
\setmainfont{DejaVu Serif}
\setsansfont{DejaVu Sans}
\setmonofont{Monaco}
\renewcommand*\familydefault{\sfdefault}
\renewcommand\CJKfamilydefault{sf}
\usetheme{Madrid} % My favorite!

\setbeamertemplate{items}[circle]
\setbeamertemplate{section in toc}{\hspace*{1em}\inserttocsection}
\setbeamertemplate{subsection in toc}{\hspace*{2em}\inserttocsubsection\\}
\setbeamertemplate{blocks}[rounded][shadow=true]
\setbeamercolor{titlelike}{parent=structure}

\setbeamertemplate{navigation symbols}{}
\setbeamertemplate{frametitle}{}

\usepackage{CJK}
\usepackage[english]{babel}
\usepackage[utf8]{inputenc}
\usepackage[T1]{fontenc}
\usepackage{helvet}
\usepackage{tikz}
\usetikzlibrary{fadings}
\usepackage{ifthen}
\usepackage{bm}     %for bold math symbols
\usepackage{xcolor} % color package; load before tocstyle
\usepackage{multicol}
\usepackage{textpos} % package for the positioning

\definecolor{secinhead}{RGB}{249,196,95}
\definecolor{shadowbg}{RGB}{51,51,51}

\definecolor{mybgcolor}{rgb}{0.361,0.412,0.463}
\definecolor{mylightblue}{rgb}{0.596,0.678,0.761}% for box (#98adc2)
\definecolor{mydarkgray}{rgb}{0.282,0.322,0.361}% for box (#48525c)
\setbeamercolor{secsubsec}{fg=mybgcolor,bg=mybgcolor}
\setbeamercolor{shadow}{fg=secinhead,bg=shadowbg}

\setbeamertemplate{headline}
{
  \leavevmode%
  \hbox{%
  \begin{beamercolorbox}[wd=\paperwidth,ht=9.25ex,dp=5.5ex,xshift=2cm]{secsubsec}%
    \raggedright
    \hspace*{2em}%
    {\sffamily\Large\color{white}
    \ifthenelse{\thesection > 0}{\thesection.~\insertsection\hfill\insertsubsection}{\insertsection\hfill\insertsubsection}}
    \hspace*{2em}%
    \ifthenelse{\thesection = 0}{\raisebox{-4.5ex}{\includegraphics[height=3cm,width=2cm,keepaspectratio]{dcu-logo}}\hspace*{1em}}{}
  \end{beamercolorbox}%
  }\vskip0pt%
  \hbox{%
  \begin{beamercolorbox}[wd=\paperwidth,ht=1ex,dp=1ex]{shadow}%
  \mbox{}
  \end{beamercolorbox}
  }%
}

\setbeamertemplate{footline}
{
  \leavevmode%
  \hbox{%
  \begin{beamercolorbox}[wd=\paperwidth,ht=6.12ex,dp=3ex]{secsubsec}%
    \raggedright
    \hspace*{2em}%
    {\sffamily\small\color{white}\insertshortauthor\hfill\thepage}%
    \hspace*{2em}%
  \end{beamercolorbox}%
  }
}
  % THIS is the line that includes the FU template!

\usepackage{arev,t1enc} % looks nicer than the standard sans-serif font
% if you experience problems, comment out the line above and change
% the documentclass option "9pt" to "10pt"

% image to be shown on the title page (without file extension, should be pdf or png)
\titleimage{title_img}

\title[我可以说中文] % (optional, use only with long paper titles)
{我可以说中文}

\subtitle
{如果有副标题的话}

\author[作者一, 作者二] % (optional, use only with lots of authors)
{第一作者 \and 第二作者}
% - Give the names in the same order as the appear in the paper.

\institute[国软] % (optional, but mostly needed)
{国际软件学院}
% - Keep it simple, no one is interested in your street address.

\date[xxxx-xx-xx] % (optional, should be abbreviation of conference name)
{Google开发者大会, 2003}
% - Either use conference name or its abbreviation.
% - Not really informative to the audience, more for people (including
%   yourself) who are reading the slides online

\subject{主题-理论计算机}
% This is only inserted into the PDF information catalog. Can be left
% out.

% you can redefine the text shown in the footline. use a combination of
% \insertshortauthor, \insertshortinstitute, \insertshorttitle, \insertshortdate, ...
\renewcommand{\footlinetext}{\insertshortinstitute, \insertshorttitle, \insertshortdate}

% Delete this, if you do not want the table of contents to pop up at
% the beginning of each subsection:
\AtBeginSubsection[]
{
  \begin{frame}<beamer>{Outline}
    \tableofcontents[currentsection,currentsubsection]
  \end{frame}
}

\begin{document}

\begin{frame}[plain]
  \titlepage
\end{frame}

\begin{frame}{大纲}
  \tableofcontents
  % You might wish to add the option [pausesections]
\end{frame}

\section{动机}

\subsection{我们了解到的基本问题}

\begin{frame}{怎么让标题信息量提高}{子标题是可选的}
  % - A title should summarize the slide in an understandable fashion
  %   for anyone how does not follow everything on the slide itself.
  \begin{itemize}
  \item
    Use \texttt{itemize} a lot.
  \item
    Use very short sentences or short phrases.
  \end{itemize}
\end{frame}

\begin{frame}{让标题透露更多信息}

  You can create overlays\dots
  \begin{itemize}
  \item using the \texttt{pause} command:
    \begin{itemize}
    \item
      First item.
      \pause
    \item
      Second item.
    \end{itemize}
  \item
    using overlay specifications:
    \begin{itemize}
    \item<3->
      First item.
    \item<4->
      Second item.
    \end{itemize}
  \item
    using the general \texttt{uncover} command:
    \begin{itemize}
      \uncover<5->{\item
        First item.}
      \uncover<6->{\item
        Second item.}
    \end{itemize}
  \end{itemize}
\end{frame}


\subsection{前人的工作}

\begin{frame}[fragile]{古老的算法}
% NB. listings is quite powerful, but not well suited to be used with beamer
%  consider using semiverbatim or the like, see below
\begin{lstlisting}[language=C]
int main (void)
{
  std::vector<bool> is_prime (100, true);
  for (int i = 2; i < 100; i++)
    if (is_prime[i])
      {
        std::cout << i << " ";
        for (int j = i; j < 100;
            is_prime [j] = false, j+=i);
      }
  return 0;
}
\end{lstlisting}
\end{frame}

\begin{frame}[fragile]
  \frametitle{An Algorithm For Finding Primes Numbers.}
\begin{semiverbatim}
\uncover<1->{\alert<0>{int main (void)}}
\uncover<1->{\alert<0>{\{}}
\uncover<1->{\alert<1>{ \alert<4>{std::}vector<bool> is_prime (100, true);}}
\uncover<1->{\alert<1>{ for (int i = 2; i < 100; i++)}}
\uncover<2->{\alert<2>{    if (is_prime[i])}}
\uncover<2->{\alert<0>{      \{}}
\uncover<3->{\alert<3>{        \alert<4>{std::}cout << i << " ";}}
\uncover<3->{\alert<3>{        for (int j = i; j < 100;}}
\uncover<3->{\alert<3>{             is_prime [j] = false, j+=i);}}
\uncover<2->{\alert<0>{      \}}}
\uncover<1->{\alert<0>{ return 0;}}
\uncover<1->{\alert<0>{\}}}
\end{semiverbatim}
  \visible<4->{Note the use of \alert{\texttt{std::}}.}
\end{frame}

\section{Our Results/Contribution}

\subsection{Main Results}

\begin{frame}{Make Titles Informative.}
  \begin{example}
    \begin{itemize}
    \item 2 is prime (two divisors: 1 and 2).
    \item 3 is prime (two divisors: 1 and 3).
    \item 4 is not prime (\alert{three} divisors: 1, 2, and 4).
    \end{itemize}
  \end{example}
\end{frame}

\begin{frame}{Make Titles Informative.}
\begin{theorem}
 There is no largest prime number and, in addition, $$\int_\Omega \nabla u \cdot \nabla v = - \int_\Omega u \Delta v + \int_{\partial\Omega} u v n$$
 \end{theorem}
 \begin{proof}
 \begin{enumerate}
 \item<1-> Suppose $p$ were the largest prime number.
 \item<2-> Let $q$ be the product of the first $p$ numbers.
 \item<3-> Then $q + 1$ is not divisible by any of them.
 \item<1-> Thus $q + 1$ is also prime and greater than $p$.\qedhere
 \end{enumerate}
 \end{proof}
 \uncover<4->{The proof used \textit{reductio ad absurdum}.}
\end{frame}

\begin{frame}{Make Titles Informative.}
\end{frame}


\subsection{Basic Ideas for Proofs/Implementation}

\begin{frame}{Make Titles Informative.}
\end{frame}

\begin{frame}{Make Titles Informative.}
\end{frame}

\begin{frame}{Make Titles Informative.}
\end{frame}



\section*{总结}

\begin{frame}{总结}

  % Keep the summary *very short*.
  \begin{itemize}
  \item
    The \alert{first main message} of your talk in one or two lines.
  \item
    The \alert{second main message} of your talk in one or two lines.
  \item
    Perhaps a \alert{third message}, but not more than that.
  \end{itemize}

  % The following outlook is optional.
  \vskip0pt plus.5fill
  \begin{itemize}
  \item
    Outlook
    \begin{itemize}
    \item
      Something you haven't solved.
    \item
      Something else you haven't solved.
    \end{itemize}
  \end{itemize}
\end{frame}



% All of the following is optional and typically not needed.
\appendix
\section<presentation>*{\appendixname}
\subsection<presentation>*{For Further Reading}

\begin{frame}[allowframebreaks]
  \frametitle<presentation>{For Further Reading}

  \begin{thebibliography}{10}

  \beamertemplatebookbibitems
  % Start with overview books.

  \bibitem{Author1990}
    A.~Author.
    \newblock {\em Handbook of Everything}.
    \newblock Some Press, 1990.


  \beamertemplatearticlebibitems
  % Followed by interesting articles. Keep the list short.

  \bibitem{Someone2000}
    S.~Someone.
    \newblock On this and that.
    \newblock {\em Journal of This and That}, 2(1):50--100,
    2000.
  \end{thebibliography}
\end{frame}

\end{document}
